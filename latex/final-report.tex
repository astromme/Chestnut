\documentclass{article}
\usepackage{fullpage}
\usepackage{apacite}
\usepackage{url}
\usepackage[bottom]{footmisc} % make sure footnote is at bottom of page
\newcommand{\comment}[1]{}

\begin{document}
\title{Chestnut: Simplifying General Purpose Graphics Processing}
\author{Andrew Stromme \& Ryan Carlson}
\date{May 10, 2010}
\maketitle

\begin{abstract}
\end{abstract}

\section{Introduction}

Say why GPGPU is cool but difficult. We have provided an easier way to deal with it. With us, you get the performance gains without the steep learning curve.

\section{Related Works}

Basically just plop our annotated bibliography in here.

\section{General Purpose Graphics Processing}

\subsection{Paradigm/Data Flow}

Data $\rightarrow$ Function $\rightarrow$ Data etc

\subsection{Tools}

There are a number of tools available to interface with the GPU. Two we have chosen are CUDA and Thrust.

\subsubsection{CUDA}

Developed by Nvidia. Allows fine-tooth control over GPU. Very verbose and C-like

\subsubsection{Thrust (RC)}

Much more C++-y. Easier to grasp. More like programming for CPU but still need to know some stuff about GPU

\section{Chestnut}

\subsection{Overview/Pipeline}

Move from Graphical Interface to Intermediate Code to Thrust code

\subsection{Goals}

Exposes Data-focused model. Modular and discoverable. Speedup over CPU (which we achieved)

\subsection{Targets/Audience}

Python Programmers. Non-CS people. Embarrassingly parallel problems.

\section{Chestnut Frontend (AS)}

\subsection{Themes (?)}

Data-centric programming model. Drag-and-Drop interface. Forces GPU programming model.

\subsection{Primitives}

DataBlocks, Values, and Functions.

\subsection{Translate GUI to Chestnut Code (RC)}

Visit each node. Each node knows its contribution. Variable Declarations and Function calls.

\section{Chestnut Backend (RC)}

\subsection{Process}

Lex tokenizes, Yacc accepts sequences of tokens. From Yacc we call helper functions which write Thrust code to disk. Compile that code using nvcc

\subsection{Primitives}

Support ints, floats. Functions: map, reduce, sort, print, read, write. Ask Tia if we should put specifications here or in an appendix or what.

\subsection{Sample Code (?)}

Do we want to show some samples? Use verbatim!

\section{Experiments (RC)}

Run both qualitative experiments -- compare code samples -- and quantitative experiments with runtimes. We show that, subjectively, our code is easier to read, if less powerful. BUT!\ runtimes are very similar.

\subsection{Qualitative}

Compare Chestnut to Thrust to CUDA

\subsection{Quantitative}

Runtimes.

\section{Future Work}

\begin{itemize}
  \item Customizable Functions and DataTypes
  \item More default Functions and DataTypes
  \item Refine GUI
  \item For-loops
\end{itemize}

\end{document}

