
\section{Introduction}

Students are often introduced to parallel programming with the concept of
heavyweight processes or threads. With this introduction students are
encouraged to think in terms of how threads communicate with each other. This
works well for concepts such as the producer-consumer problem and for
parent-child models but a more simple parallel model is sufficient for a wide
array of problem domains.

We observe that often the same operation is performed to each element in a
large array of data. This can be seen from simple calculations (add 4 to each
element of an array) to more complex situations such as generating an updated
array value by some combination of the neighboring points (this is used in
simulations such as heat flow models). In traditional sequential
implementations of these computations there is a core operation that is
performed many times within one or more loops. The entire computation can be
parallelized with a simple parallel model when updating each element can be
performed independently of updating any other element.

The Chestnut parallel model assumes that a program exists as sequential code
with small self-contained blocks of parallel code interspersed throughout the
program. Parallel blocks, or contexts, exist to update arrays with new values
in parallel. When declarating a parallel context the programmer specifices a
foreach loop based on some array; one iteration of the loop is run for each
element in the array. Each iteration of this loop is performed in a separate
thread in parallel with all other iterations. Each thread has access to the
state of all arrays as they were before the loop started. This means that even
if one thread modifies a value in an array within a parallel context all other
threads within this context can safely use the same location in the array; they
will get the old values instead. There is an implicit synchronization barrier
at the end of each parallel context where all threads must finish their
computations before the sequential code following the parallel contex can
resume. After the parallel context the values of any arrays that have been
modified now reflect the new values rather than the pre-parallel-context
values. Subsequent array accesses get the new values.

This parallel model fits the graphics card well--typically parallel contexts
will perform the same operation across all threads, a condition which is needed
for the processors in a graphics card to be fully utilized.

A difficult part with any parallel computation is the specification of the
parallelism. In general Chestnut programs will specify their parallelism based
on the output arrays--one computation is one for each element in the output
array, combining the necessary inputs. For an example matrix multiplication
application the foreach loop would iterate over the output array and each
thread would access the necessary row and column of the input arrays to perform
the computations necessary to compute that output value. 

Talk about chestnut being research into a simplification of the GPU general
purpose programming model. Locality is important, simple programs are simple.
Simulations, visualizations. Visualizations stay on the GPU, and pixel
rasterization uses the same language.


\section{Background}

\begin{itemize}
\item Background section about CUDA
\item Background about GPU Computing
\end{itemize}

\section{Related Work}

\subsection{Educational Programming Languages}

\begin{itemize}
\item Scratch
\end{itemize}

\subsection{Existing Layers on top of CUDA}
\begin{itemize}
\item Mint
\item Stencil programming
\item HiCUDA
\item Physis
\end{itemize}

\subsection{Scientific Programming and Simulations}

\begin{itemize}
\item Physics work from Amy Bug?
\item Diffusion
\item Himeno
\end{itemize}


\section{Chestnut}

Chestnut is presented as a framework for parallel computing. The Chestnut
language is C-style imperative language that can be used to write sequential
code interspersed with parallel contexts. This language can be compiled to C++
source code with the Chestnut compiler. This compiler performs source to source
translations and makes heavy use of the Walnut library. Walnut is a C++ library
which is built on top of CUDA and Thrust and is used to provide a Chestnut-like
API at the C++ level. Currently Walnut is designed to be used internally by the
Chestnut compiler but future work could expand its reach to a normal C++
library to be used directly by programmers. One the top of all of this
infrastructure is the Chestnut designer which is a gui programming tool that
exposes the power of the Chestnut parallel model and uses a drag and drop
interface.

Description of the parts of chestnut including the core language syntax, the
paradigms, the chestnut designer gui and the chestnut compiler implementation
(with its underlying Walnut library).

\subsection{Chestnut Language}

%$[parallel|sequential] ReturnType function_name(Type param1, Type param2) { expressions }$
%$$

\subsubsection{The Parallel Context}

\begin{itemize}
\item Reference the parts of the Introduction which discuss the parallel model.  
\item Example-driven explanation of a parallel context
\item Describe the particular semantics of a parallel context (no sequential functions, no nested contexts, etc)
\end{itemize}

\subsubsection{Sequential Functions}

\begin{itemize}
\item Simple C-style sequential functions.
\item Can access array locations (in a slow manner)
\item Can contain parallel contexts
\item CPU-Based
\end{itemize}

\subsubsection{Parallel Functions}

\begin{itemize}
\item Very similar to sequential functions
\item Can't contain parallel contexts
\item GPU-Based
\end{itemize}

\subsection{Chestnut Designer}

\begin{itemize}
\item Drag and drop interface
\item Cross platform, written in Qt
\item Can only do things that actually make sense
\item sort of Dataflow-like
\end{itemize}

\subsection{Visualizations}

\begin{itemize}
\item GPU-Driven visuals
\item Uses the same parallel model
\item Everything stays on the GPU
\item very basic model
\end{itemize}

\subsection{Chestnut Compiler}

Frontend

\subsection{Translating Chestnut Code to CUDA C}

Backend. Examples. Comparison of Chestnut and CUDA C side-by-side.

Here is how a foreach with a window works
Here is how a foreach with a element access works
Here is how...

%\begin{figure}[htb]
%\begin{center}
%\includegraphics[width=40mm]{mouthDetected.png}
%\caption{Mouth Detected on Typical Face}
%\label{fig:mouthDetected}
%\end{center}
%\end{figure}
   

\section{Results}

graphs? Maybe merged into the next section?

%\begin{equation}
%\% \text{Error} = \frac{\text{False Positives} + \text{False Negatives}}{\text{False Positives} + \text{False Negatives} + \text{True Positives} + \text{True Negatives}} = 22.5\%
%\end{equation}



\section{Comparing Chestnut Programs to CUDA and Sequential C Programs}



\section{Future Work}

Work on the desginer. Work on supporting some of the missing functionality.
Language for scientific programming vs language for learning about parallel
programming. Support for OpenCL backends (\& AMD GPUs)



\section{Conclusion}

Works. Is Faster. Is interesting. Blah
